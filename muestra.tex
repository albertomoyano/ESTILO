\documentclass[11pt]{article}
\usepackage[textwidth=5.5in,textheight=8.5in]{geometry}
\usepackage[T1]{fontenc}
\usepackage[scale=0.94]{plex-serif}
\usepackage[]{plex-sans}
\usepackage[scale=0.91]{plex-mono}
\usepackage{lettrine}
\renewcommand{\ttfamily}{\fontencoding{OT1}\fontfamily{cmtt}\selectfont}
\PassOptionsToPackage{urlcolor=black,colorlinks}{hyperref}
\RequirePackage{hyperref}
\usepackage{xcolor}
\usepackage{longtable}
\usepackage{multicol}
\newcommand{\typeone}{Type~\liningnums{1}}
\begin{document}
\title{\LaTeX\ Support for Linux Libertine and Biolinum Fonts}
\author{Bob Tennent\\
\small\url{rdt@cs.queensu.ca}}
\date{\today}
\maketitle 
\thispagestyle{empty}
\begin{small}
\tableofcontents
\end{small}
\sloppy
\clearpage
\section{Introduction}

This package provides support for use of the Linux Libertine and Linux
Biolinum families of fonts in \LaTeX.
Most features 
are usable
with \LaTeX\ and \verb|dvips|, pdf\LaTeX, xe\LaTeX\ and lua\LaTeX;
the features in Section~\ref{OpenType} are only usable with xe\LaTeX\ or lua\LaTeX. 
This package compatibly replaces several earlier packages (\texttt{libertine-type1},
\texttt{biolinum-type1}, \texttt{libertine}) and should provide partial compatibility with
the obsolete \texttt{libertineotf} and \texttt{libertine-legacy} packages.

\section{Installation}

To install this package on a TDS-compliant \TeX\ system, download the
file 
\begin{list}{}{}\item \verb\tex-archive/install/fonts/libertine.tds.zip\ 
\end{list}
and unzip at the root of an appropriate
\verb\texmf\ tree, likely a personal or local tree. If necessary, update
the file-name database (e.g., \verb\texhash texmf\). Update the font-map files by
enabling the Map file \verb\libertine.map\.


\section{Basic Usage}

For most purposes, simply add

\begin{list}{}{}\item
\verb|\usepackage{libertine}|
\end{list}
to the preamble of your document. This will activate Libertine as
the main (seriffed) text font, Biolinum as the sans font,
and (from January~2013) LibertineMono as the monospaced font. 
It is
recommended that the font encoding be set to \verb\T1\ or \verb\LY1\ but the default
\verb\OT1\  encoding is also supported. Available shapes in all series (except \texttt{tt}, which
only has \texttt{it}) include:
\begin{list}{}{}\item
\begin{tabular}{ll}
\texttt{it} &               italic\\
\texttt{sc} &            small caps\\
\texttt{scit} &            italic small caps
\end{tabular}
\end{list}
Slanted variants are not supported; the designed italic variants will be
automatically substituted. The exceptions are the monospaced font and the bold series of Biolinum,
for which designed italics are not currently available. Artificially
slanted variants have been generated and treated as if they were italic.

To activate Libertine (without Biolinum), use the \texttt{libertine} (or \texttt{rm})
option. Similarly, to activate Biolinum (without Libertine) use the
\texttt{biolinum} (or \texttt{sf} or \texttt{ss}) option. To use Biolinum as the main text font (as
well as the sans font), use the option \texttt{sfdefault}.
Use the \verb|mono=false| (or \verb|tt=false|) option to suppress
activating LibertineMono. 
To activate single font families,
use one or more of
\begin{list}{}{}
\item \verb|\usepackage{libertineRoman}|
\item \verb|\usepackage{libertineMono}|
\item \verb|\usepackage{biolinum}|
\end{list}


\section{Advanced Usage}

Lua\LaTeX\ and xe\LaTeX\ users who might prefer to use Type~1 fonts or who
wish to avoid \texttt{fontspec} may use the \texttt{type1} (or \texttt{nofontspec}) option.
The \verb\libertine-type1.sty\,
\verb\biolinum-type1.sty\ and \verb\libertineMono-type1.sty\ packages provide
compatibility
with older packages. For legacy documents that use only basic
facilities of \verb\libertineotf\, a wrapper package \verb\libertineotf.sty\ is provided.
The following features of the original \verb|libertine| or \verb|libertineotf|
packages are
\emph{not} supported:
\begin{itemize}

 \item font-features such as \texttt{Ligatures} or \texttt{Scale} as option parameters

 \item the Outline or Shadow fonts 

 \item commands \verb|\Lnnum|, \verb|\Lpnum|, \verb|\Lcnum|, etc.

 \item environments \texttt{Ltable} and \texttt{libertineenumerate} 
\end{itemize}
If your documents use any of the features listed above,
you may have to continue to use the \verb\libertineotf\ package (which is still available from
CTAN) or access the OpenType fonts directly using \texttt{fontspec}.


The following options are available in all styles (except monospaced):
\begin{list}{}{}\item
\begin{tabular}{ll}
\texttt{oldstyle} (\texttt{osf}) &  old-style figures\\
\texttt{lining} (\texttt{nf}, \texttt{lf}) &  lining figures\\
\texttt{proportional} (\texttt{p})&  varying-width figures\\
\texttt{tabular} (\texttt{t}) &     fixed-width figures\\
\end{tabular}
\end{list}
The defaults (from January~2013) are \texttt{lining} and \texttt{tabular}. These apply to both Libertine
and Biolinum;
to change the default figure style of just the Biolinum (sans)
fonts, use options
\begin{list}{}{}\item
\texttt{sflining} (\texttt{sflf}) or \texttt{sfoldstyle} (\texttt{sfosf}, \texttt{osfss})
\item
\texttt{sftabular} (\texttt{sft}) or \texttt{sfproportional} (\texttt{sfp}) 
\end{list}

The \texttt{semibold} (\texttt{sb}) option will enable use of the
semi-bold series of Libertine; this has no effect on the Biolinum fonts,
for which there is no semi-bold variant. The options \verb|scale=|<\emph{number}> (or
\verb|scaled=|<\emph{number}>) will scale the Biolinum fonts but have no effect on the
Libertine fonts. Similarly, the options 
\verb|llscale=|<\emph{number}> (or  \verb|llscaled=|<\emph{number}>)
and
\verb|ttscale=|<\emph{number}> (or  \verb|ttscaled=|<\emph{number}>)
will scale the LinuxLibertine and LibertineMono fonts, respectively.  
Any of the ``Boolean'' options, such as \texttt{osf}, may 
also be used
in the form \verb|osf=true| or \verb|osf=false|.

The option \verb\defaultfeatures=...\ allows the user to add default OpenType
features; for example, \verb\defaultfeatures={Variant=01}\ will force use of the Stylistic~Set~1
variant glyphs.

Commands \verb|\oldstylenums{|\ldots\verb|}| and \verb|\oldstylenumsf{|\ldots\verb|}| are defined to
allow for local use of old-style figures in Libertine and Biolinum,
respectively, if lining figures is the default, and similarly
\verb|\liningnums{|\ldots\verb|}| and \verb|\liningnumsf{|\ldots\verb|}|.

Similarly, commands \verb|\tabularnums{|\ldots\verb|}| and \verb|\tabularnumsf{|\ldots\verb|}| are defined
to allow local use of monospaced figures in Libertine or Biolinum,
respectively, if proportional figures is the default, and similarly
\verb|\proportionalnums{|\ldots\verb|}| and \verb|\proportionalnumsf{|\ldots\verb|}|.

Superior numbers (for footnote markers) are available using \verb|\sufigures|
or \verb|\textsu{|\ldots\verb|}|.


Command \verb|\useosf| switches the default figure style for Libertine and Biolinum to old-style figures; this is
primarily for use \emph{after} calling a math package (such as \verb|newtxmath| with the
\verb|libertine| option) with lining figures as the default.

The following macros select the font family indicated:
\begin{center}
\begin{tabular}{ll}
\verb|\libertine| & Libertine \\
\verb|\libertineSB|& Libertine with semibold \\
\verb|\libertineOsF| & Libertine with oldstyle figures \\
\verb|\libertineLF| & Libertine with lining figures \\
\verb|\libertineDisplay| & Libertine Display \\
\verb|\libmono| & Libertine Monospaced \\
\verb|\libertineInitial| & Libertine Initials \\
\verb|\biolinum| & Biolinum \\
\verb|\biolinumOsF|& Biolinum with oldstyle figures \\
\verb|\biolinumLF| & Biolinum with lining figures \\
\end{tabular}
\end{center}
Macro \verb|\libertineInitialGlyph{|\ldots\verb|}| produces a glyph in the Libertine Initial font;
Appendix~\ref{InitialGlyphs} has a table of some of the glyphs.

\section{OpenType Fonts}
\label{OpenType}

The features in this section are only available to xe\LaTeX\ and lua\LaTeX\ users.

Macros \verb|\libertineGlyph{|\ldots\verb|}| and \verb|\biolinumGlyph{|\ldots\verb|}| produce the
glyph named in the argument in the Libertine or Biolinum font,
respectively; for example, in regular-weight and upright-shape,
\verb|\libertineGlyph{seven.cap}| and \verb|\libertineGlyph{uniE10F}| both produce a
lining~7 that matches the height of capital letters, as in
\begin{list}{}{}\item
% K\libertineGlyph{seven.cap}L~\libertineGlyph{three.cap}N\libertineGlyph{six.cap}
\end{list} 
Similarly, \verb|\biolinumKeyGlyph{|\ldots\verb|}| produces the named glyph
% in the Biolinum Keyboard font; for example: \verb|\biolinumKeyGlyph{seven}| produces \biolinumKeyGlyph{seven}.
A large number of macros of the form \verb|\LKey|\ldots or \verb|\LMouse|\ldots
are provided to simplify production of glyphs in the Biolinum Keyboard font;
see Appendix~\ref{LKey}. Appendix~\ref{KeyboardGlyphs} has a table of the entire
Linux Biolinum Keyboard font, with corresponding glyph name and codepoint.


The directory
\verb|/fonts/opentype/public/libertine| 
has the fonts used for these features, as follows:
\begin{list}{}{}\item\small
\begin{tabular}{lll}
\multicolumn{1}{c}{\bf File name} & \multicolumn{1}{c}{\bf Internal name} & \multicolumn{1}{c}{\bf Description} \\
\hline
\verb|LinBiolinum_RBO.otf|  & \verb|LinBiolinumOBO|  & sans serif bold italic (oblique) \\
\verb|LinBiolinum_RB.otf|  & \verb|LinBiolinumOB| & sans serif bold \\
\verb|LinBiolinum_RI.otf|  & \verb|LinBiolinumOI| & sans serif italic \\
\verb|LinBiolinum_R.otf|  & \verb|LinBiolinumO| & sans serif regular \\
\verb|LinLibertine_RBI.otf|  & \verb|LinLibertineOBI| & bold italic \\
\verb|LinLibertine_RB.otf|  & \verb|LinLibertineOB| & bold \\
\verb|LinLibertine_RI.otf|  & \verb|LinLibertineOI| & italic \\
\verb|LinLibertine_R.otf|  & \verb|LinLibertineO| & regular \\
\verb|LinLibertine_RZI.otf|  & \verb|LinLibertineOZI| & semibold italic \\
\verb|LinLibertine_RZ.otf|  & \verb|LinLibertineOZ| & semibold \\
\verb|LinLibertine_MBO.otf| & \verb|LinLibertineMOBO| & mono bold italic (oblique) \\
\verb|LinLibertine_MB.otf| & \verb|LinLibertineMOB| & mono bold \\
\verb|LinLibertine_MO.otf| & \verb|LinLibertineMOO| & mono italic (oblique) \\
\verb|LinLibertine_M.otf| & \verb|LinLibertineMO| & mono \\
\verb|LinBiolinum_K.otf|  & \verb|LinBiolinumOKb| & keyboard \\
\verb|LinLibertine_I.otf|  & \verb|LinLibertineIO| &   decorative capitals \\
\verb|LinLibertine_DR.otf| & \verb|LinLibertineDisplayO| &   a display (titling) font  \\
\end{tabular}
\end{list}

\section{Concluding Remarks}


For compatible mathematics, it is recommended to use
\begin{verbatim}
  \usepackage[libertine]{newtxmath}
\end{verbatim}
with pdf\LaTeX\ and
\begin{verbatim}
  \usepackage{unicode-math}
  \setmathfont[Scale=MatchUppercase]{libertinusmath-regular.otf}
\end{verbatim}
with xe\LaTeX\ or lua\LaTeX.

The original OpenType fonts were created by Philipp H. Poll 
(\url{gillian@linuxlibertine.org}) and are licensed under the terms of the GNU General
Public License (Version~2, with font exception) and under the terms of
the Open Font License. For details look into the \verb|doc| directory of the
distribution or at
\begin{list}{}{}\item
\url{http://www.linuxlibertine.org/}
\end{list}
The Glyph and KeyCap support was adapted from the original \verb\libertine\
package by Michael Niedermair.

Three of the Libertine fonts were modified by Michael Sharpe (\url{msharpe@ucsd.edu}) using 
\texttt{fontforge} to correct minor problems, including adding
three missing ligatures (\emph{\bfseries fl, ffl, ffi}) to the bold-italic font.

% The \typeone\ fonts were created using \verb|cfftot1| or \verb|fontforge|. 
% The internal font-family names of the \typeone\ 
fonts have been changed to \verb|Linux Libertine T| and \verb|Linux Biolinum T| to avoid
interfering with xe\LaTeX\ users who access system fonts. 

The support
files were created using \verb|autoinst|. The support files are licensed under
the terms of the LaTeX Project Public License.  
See Appendix~\ref{impl} for more detailed discussion of the implementation.

Thanks to Herbert Voss, Patrick Gundlach, Silke Hofstra, Marc Penninga, Michael Sharpe,
Denis Bitouz\'{e}, and Khaled Hosny for their assistance.
The maintainer of this
package is Bob Tennent (\url{rdt@cs.queensu.ca})



\clearpage
\section{Linux Biolinum Keyboard Glyphs}\small\tt
\label{KeyboardGlyphs}
\renewcommand\DeclareTextGlyphY[3]{\makebox[2.5cm]{\LARGE\strut\fbox{\biolinumKeyGlyph{#2}}} #2\\}%
\catcode`\_=12%
\begin{multicols}{2}
\par\noindent
\input{LinBiolinum_K}
\end{multicols}


\end{document}
